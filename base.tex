\usepackage{fontspec}
\usepackage{mathtools}
\usepackage{unicode-math}
\usepackage{tikz}
\usetikzlibrary{trees}
\usepackage{mathpartir}
\usepackage{parskip}
\usepackage{minted}
\usepackage{csquotes}
\usepackage{tcolorbox}
\usepackage{biblatex} % TODO style w/o quotation marks
\usepackage{siunitx}
\usepackage{booktabs}
\usepackage{amsmath,amssymb}
\usepackage{color,soul}
\usepackage{xcolor}

\def\mul#1#2{\color{#1}\underline{{\color{black}#2}}\color{black}}

\newtheorem{lem}{Lemma}

\setmainfont{XITS}
\setmathfont{XITS Math}
\setsansfont[Scale=MatchLowercase]{DejaVu Sans}
\setmonofont[Scale=MatchLowercase]{DejaVu Sans Mono}

\addbibresource{lit.bib}

% Emoji images are from
% https://github.com/joypixels/emoji-assets/blob/master/png/128
\newlength{\emojiheight}
\settoheight{\emojiheight}{H}
\newcommand{\good}{\includegraphics[height=\emojiheight]{images/1f973}}
\newcommand{\bad}{\includegraphics[height=\emojiheight]{images/1fae4}}

\setbeamertemplate{navigation symbols}{}
\setbeamertemplate{itemize item}{\Large\textbullet}
\setbeamertemplate{itemize subitem}{\Large\textbullet}
\setbeamertemplate{itemize subsubitem}{\Large\textbullet}

\tcbset{frame empty}

\setminted[lean4]{extrakeywords={aesop cases add aesop? intro simp simp_all only split apply on_goal next rename_i safe unsafe norm constructors forward destruct norm_num done add_aesop_rules rfl subst ext}}
\newmintinline[lean]{lean4}{bgcolor={},ignorelexererrors=true}
\newminted[leancode]{lean4}{bgcolor={},ignorelexererrors=true,fontsize=\footnotesize,autogobble}
\BeforeBeginEnvironment{leancode}{\begin{tcolorbox}}
\AfterEndEnvironment{leancode}{\end{tcolorbox}}
\usemintedstyle{xcode}

\setlength{\parskip}{1em}
\setlength{\tabcolsep}{0.5em}

% Source: https://tex.stackexchange.com/questions/55806/mindmap-tikzpicture-in-beamer-reveal-step-by-step/55849#55849
%
% Keys to support piece-wise uncovering of elements in TikZ pictures:
% \node[visible on=<2->](foo){Foo}
% \node[visible on=<{2,4}>](bar){Bar}   % put braces around comma expressions
%
% Internally works by setting opacity=0 when invisible, which has the
% advantage (compared to \node<2->(foo){Foo} that the node is always there, hence
% always consumes space plus that coordinate (foo) is always available.
%
% The actual command that implements the invisibility can be overridden
% by altering the style invisible. For instance \tikzsset{invisible/.style={opacity=0.2}}
% would dim the "invisible" parts. Alternatively, the color might be set to white, if the
% output driver does not support transparencies (e.g., PS)
\tikzset{
  invisible/.style={opacity=0},
  visible on/.style={alt={#1{}{invisible}}},
  alt/.code args={<#1>#2#3}{%
    \alt<#1>{\pgfkeysalso{#2}}{\pgfkeysalso{#3}} % \pgfkeysalso doesn't change the path
  },
}

% Source: https://tex.stackexchange.com/questions/102069/make-a-heading-in-beamer
\newcommand\frameheading[1]{%
  \par\bigskip
  {\Large\usebeamercolor[fg]{palette primary}#1}\par\smallskip}

% Colourblind-friendly palette from https://www.nceas.ucsb.edu/sites/default/files/2022-06/Colorblind%20Safe%20Color%20Schemes.pdf
\definecolor{csgreen}{RGB}{000,158,115}
\definecolor{csorange}{RGB}{213,094,000}
\definecolor{csblue}{RGB}{000,114,178}
\definecolor{cspurple}{RGB}{204,121,167}
\newcommand{\orange}[1]{\textcolor{csorange}{#1}}
\newcommand{\blue}[1]{\textcolor{csblue}{#1}}
\newcommand{\grey}[1]{\textcolor{lightgrey}{#1}}
\newcommand{\green}[1]{\textcolor{csgreen}{#1}}
\newcommand{\purple}[1]{\textcolor{cspurple}{#1}}

\setbeamercolor{structure}{fg=csblue}

\renewcommand{\iff}{\leftrightarrow}
\newcommand{\com}{,\,}
\renewcommand{\emph}[1]{\blue{#1}}
\newcommand{\mv}[1]{\ensuremath{\mathit{?#1}}}
\newcommand{\rulename}[1]{\textrm{#1}}
\newcommand{\rulelabel}[1]{\quad \text{\rulename{#1}}}
\newcommand{\Expr}{\ensuremath{\mathbb{E}}}
\newcommand{\Hyps}{\ensuremath{\mathbb{H}}}
\newcommand{\Matches}{\ensuremath{\mathbb{M}}}
\newcommand{\Slots}{\ensuremath{\mathbb{S}}}
\newcommand{\Vars}{\ensuremath{\mathbb{V}}}
\newcommand{\Pow}[1]{\ensuremath{\mathcal{P}(#1)}}
\newcommand{\dom}[1]{\ensuremath{\mathrm{dom}(#1)}}
\newcommand{\cod}[1]{\ensuremath{\mathrm{cod}(#1)}}
\newenvironment{rapppic}{\begin{tikzpicture}[outer sep=auto, level distance=3em]}{\end{tikzpicture}}
\newenvironment{rapp}{%
  \begin{tcolorbox}
  \begin{center}
  \begin{rapppic}
}{
  \end{rapppic}
  \end{center}
  \end{tcolorbox}%
}

\begin{document}

\title{Efficient Forward Reasoning for Aesop}
\author{Xavier Généreux \qquad Jannis Limperg\\ University of Munich (LMU)}
\date{January 14, 2025}

\setbeamertemplate{footline}[text line]{\parbox{\textwidth}{\centering \url{https://github.com/JLimperg/talk-2025-01-lean-together-aesop-forward}\medskip}}

\begin{frame}
  \maketitle
\end{frame}

\setbeamertemplate{footline}{}

\AtBeginSection{
  \begin{frame}
    \usebeamercolor[fg]{frametitle}
    \Large \center{\insertsection}
  \end{frame}
}

%\begin{frame}
%  \tableofcontents
%\end{frame}

\section{Forward Reasoning with Aesop}

% Beginning of Xavier's frames
\begin{frame}
  \frametitle{Forward and Backward Reasoning}

  \begin{itemize}
    \item Backward:
    To prove a goal, we reduce it to simpler goals.
    \begin{itemize}
      \item In Lean: \lean{apply}
      \item Works on a goal's target.
    \end{itemize}

    \item Forward:
    From given facts, we deduce new facts.
    \begin{itemize}
      \item In Lean: \lean{have} and \lean{obtain}
      \item Works on a goal's hypotheses.
    \end{itemize}
  \end{itemize}
\end{frame}

\begin{frame}
  \frametitle{Aesop}

  \begin{itemize}
    \item Tactic for customisable proof search.
    \item Applies user-specified rules to goals.
    \item Rules are arbitrary tactics.
    \item If a rule does not fully solve a goal, it produces subgoals.
    \item Aesop proves the subgoals recursively.
  \end{itemize}
\end{frame}

%\begin{frame}
%  \frametitle{Forward Reasoning}
%
%  \begin{itemize}
%    \item Isabelle
%    \item Mathematics
%  \end{itemize}
%\end{frame}


%\begin{frame}
%  \frametitle{Backward Reasoning}
%
%  \begin{itemize}
%    \item Lean$^*$
%  \end{itemize}
%\end{frame}

\begin{frame}
  \frametitle{Aesop Rules}

  Aesop provides convenient syntax (rule builders) for
  creating rules from theorems.

  \pause

  \begin{itemize}
    \item Rules tagged with \lean{apply} try to use the rule backwards.
    \item Rules tagged with \lean{forward} try to use the rule forwards.
  \end{itemize}
\end{frame}

%\begin{frame}
%  \frametitle{Aesop rules (\lean{apply})}
%  \begin{block}{Note: }
%    You can tag a rule as \lean{safe} if you believe that, given a goal
%    that matches the result of the rule, \textbf{it is always the right move to use.}
%    Otherwise, you can tag a rule as \lean{unsafe} so that \lean{aesop} can backtrack.
%    \end{block}
%
%\end{frame}

\begin{frame}
  \frametitle{What Is A Good Backward Rule?}
  \begin{lem}[\texttt{And.intro}]
    If the propositions 
    \only<1,3>{$A$ and $B$}\setulcolor{csorange}\only<2>{\ul{$A$} and \ul{$B$}}
    are true, then so is
    \only<1,2>{$A \land B$}\setulcolor{csblue}\only<3>{\ul{$A \land B$}.}
  \end{lem}
\end{frame}


\begin{frame}
  \frametitle{What Is A Good Forward Rule?}
  \begin{lem}[\texttt{le\_trans}]
    Given a preorder on some type $\alpha$ and elements $a,b,c$ of type $\alpha$, the following holds:
    \only<1>{$$a \leq b \rightarrow b \leq c \rightarrow a \leq c.$$}
    \only<2>{$$a \leq b \rightarrow b \leq c \rightarrow \mul{csorange}{a \leq c}.$$}
    \only<3>{$$\mul{csblue}{a \leq {\color{csblue}{b}} \rightarrow {\color{csblue}{b}} \leq c} \rightarrow a \leq c.$$}
  \end{lem}

\end{frame}

\begin{frame}
  \frametitle{What Is A Good Forward Rule?}
  \begin{lem}[\texttt{cauchySeq\_tendsto\_of\_complete}]
   
    \only<1>{Let $(a_n)$ be a Cauchy sequence in a complete space $X$.
    
    The sequence $(a_n)$ converges.
    }
    \only<2>{Let $(a_n)$ be a Cauchy sequence in a complete space $X$.
    
    
    \setulcolor{csorange}
    The sequence \ul{$(a_n)$ converges}.
    }
    \only<3>{
      \setulcolor{csblue}
    Let $(a_n)$ be a \ul{Cauchy} sequence in a \ul{complete} space $X$.
    
    The sequence $(a_n)$ converges.
    }
    

    
  \end{lem}
\end{frame}

\begin{frame}
  \frametitle{What Is A Good Forward Rule?}
  \begin{lem}[\texttt{IsClosed.isSeqClosed}]
    
    \only<1>{
    Let $s$ be a closed set in a $\mathbb{R}$ and
    $(a_n) \to a$ be converging sequence such that for all $n, a_n \in s$.
    
    Then $a \in s$.}
    \only<2>{
      Let $s$ be a closed set in a $\mathbb{R}$ and
      $(a_n) \to a$ be converging sequence such that for all $n, a_n \in s$.
      
      \setulcolor{csorange}
      Then \ul{$a \in s$.}}
    \only<3>{
      \setulcolor{csblue}
    Let $s$ be a \ul{closed} set in a $\mathbb{R}$ and
    $(a_n) \to a$ be \ul{converging} sequence such that \ul{for all $n, a_n \in s$}.
    
    Then $a \in s$.}

    
  \end{lem}
\end{frame}

\begin{frame}
  \frametitle{Disclaimer: Power of Forward Rules}
  Backward rules \enquote*{try harder}:
  \begin{itemize}
    \item<2-> Forward rules {\orange{will not}} succeed if one of the premises is missing from the context.
    \item<3-> Backward rules {\blue{may}} succeed as they also try to prove the premises.
  \end{itemize}

  However, forward rules can also be configured to leave certain premises to be proved later (with the \texttt{immediate} option).
\end{frame}


% End of Xavier's frames


\section{Efficient Implementation}

\begin{frame}[fragile]
  \frametitle{Running Example}

  \begin{leancode}
    axiom R : ℕ → ℕ → Prop

    @[aesop safe forward]
    axiom trans : ∀ x y z, R x y → R y z → R x z

    example
        (h₁ : R 300 301)
        (h₂ : R 301 302)
        (h₃ : R 302 303) :
        R 300 303 := by
      aesop
  \end{leancode}
\end{frame}

\begin{frame}[fragile]
  \frametitle{Naive Algorithm}

  \begin{leancode}
    trans : ∀ x y z, R x y → R y z → R x z

    h₁ : R 300 301
    h₂ : R 301 302
    h₃ : R 302 303
  \end{leancode}

  \begin{enumerate}[<+->]
    \item $R~y~z ≟ R~300~301 ⇒ \{y ↦ 300, z ↦ 301\}$ \good
          \begin{enumerate}[<+->]
            \item $R~x~y ≟ R~300~301 ⇒ \{x ↦ 300, y ↦ 301\}$ \bad
            \item $R~x~y ≟ R~301~302 ⇒ \{x ↦ 301, y ↦ 302\}$ \bad
            \item $R~x~y ≟ R~302~303 ⇒ \{x ↦ 302, y ↦ 303\}$ \bad
          \end{enumerate}
    \item $R~y~z ≟ R~301~302 ⇒ \{y ↦ 301, z ↦ 302\}$ \good
          \begin{enumerate}[<+->]
            \item $R~x~y ≟ R~300~301 ⇒ \{x ↦ 300, y ↦ 301\}$ \good \\
                  \orange{new hypothesis:} $h_{4} : R~300~302$
          \end{enumerate}
  \end{enumerate}
\end{frame}

\begin{frame}[fragile]
  \frametitle{Naive Algorithm}

  \begin{block}{New goal}
    \begin{leancode}
        (h₁ : R 300 301)
        (h₂ : R 301 302)
        (h₃ : R 302 303)
        (h₄ : R 300 302) :
        ⊢ R 300 303
    \end{leancode}
  \end{block}
\end{frame}

\begin{frame}
  \begin{enumerate}[<+->]
    \item $R~y~z ≟ R~300~301 ⇒ \{y ↦ 300, z ↦ 301\}$ \good
          \begin{enumerate}[<+->]
            \item $R~x~y ≟ R~300~301 ⇒ \{x ↦ 300, y ↦ 301\}$ \bad
            \item \dots
          \end{enumerate}
    \item $R~y~z ≟ R~301~302 ⇒ \{y ↦ 301, z ↦ 302\}$ \good
          \begin{enumerate}[<+->]
            \item $R~x~y ≟ R~300~301 ⇒ \{x ↦ 300, y ↦ 301\}$ \orange{already exists}
            \item $R~x~y ≟ R~301~302 ⇒ \{x ↦ 301, y ↦ 302\}$ \bad
            \item \dots
          \end{enumerate}
    \item $R~y~z ≟ R~302~303 ⇒ \{y ↦ 302, z ↦ 303\}$ \good
          \begin{enumerate}
            \item \dots
          \end{enumerate}
  \end{enumerate}
\end{frame}

\begin{frame}
  \frametitle{Central Issue}

  The naive algorithm is \orange{stateless}.

  This leads to lots of \orange{duplicate work}.

  \pause

  \frameheading{Solution}

  \blue{Cache} work done for one goal in a \blue{forward reasoning state}.

  \blue{Reuse} this state for subgoals.
\end{frame}

\begin{frame}[fragile]
  \frametitle{Stateful Algorithm Interface}

  \begin{leancode}
    addHyp   : State → Hyp → State × List NewHyp
    eraseHyp : State → Hyp → State
  \end{leancode}
\end{frame}

\begin{frame}[fragile]
  \frametitle{Rule Terminology}

  \begin{leancode}
    trans : ∀ x y z (p₁ : R x y) (p₂ : R y z), R x z
  \end{leancode}

  $x, y, z$ are determined by a match for $p₁$ and $p₂$.

  $p₁$ and $p₂$ are the \emph{slots} of \texttt{trans}.

  They are connected by the \emph{shared variable} $y$.
\end{frame}

\begin{frame}[fragile]
  \frametitle{Matches}

  \begin{leancode}
    trans : ∀ x y z (p₁ : R x y) (p₂ : R y z), R x z
  \end{leancode}

  A \emph{complete match} for \texttt{trans} is a substitution
  \[
    \{1 ↦ t, 2 ↦ u\}
  \]
  such that
  \begin{itemize}
    \item $t : R~a~b$ (matching $p₁$)
    \item $u : R~c~d$ (matching $p₂$)
    \item $b = c$ (since $y$ is shared)
  \end{itemize}

  \pause

  A \emph{partial match} is a substitution for some nonempty prefix of the slots.
\end{frame}

\begin{frame}
  \frametitle{Rule State}

  The \emph{rule state} for \texttt{trans} is a pair of maps
  \begin{align*}
    H(s, v, e) &↦ \{h₁, \dots, hₖ\} \\
    M(s, v, e) &↦ \{m₁, \dots, mₗ\}
  \end{align*}
  where
  \begin{itemize}
    \item $s$ is a slot index of \texttt{trans} (i.e.\ $s ∈ \{1,2\}$)
    \item $v$ is a shared variable (i.e.\ $v ∈ \{y\}$)
    \item $e$ is a Lean expression
    \item $h₁, \dots, hₖ$ are hypotheses
    \item $m₁, \dots, mₗ$ are matches
  \end{itemize}
\end{frame}

\begin{frame}[fragile]
  \frametitle{Adding a Hypothesis: Example}

  \begin{leancode}
    trans : R x y → R y z → R x z

    h₂ : R 301 302
    h₁ : R 300 301
  \end{leancode}

  \begin{tikzpicture}[outer sep=auto, level distance=10mm]
    \node {$H$}
      child {node[xshift=-.3cm] {1}
        child[] {node {$y$}
          child[visible on=<2->] {node[xshift=.3cm] {$302$}
            child {node {$\{h₂\}$}}}
          child[visible on=<5->] {node[xshift=-.3cm] {$301$}
            child {node {$\{h₁\}$}}}}}
      child {node[xshift=.3cm] {2}
        child {node {$y$}
          child[visible on=<4->] {node[xshift=.3cm] {$301$}
            child {node {$\{h₂\}$}}}
          child[visible on=<8->] {node[xshift=-.3cm] {$300$}
            child {node {$\{h₁\}$}}}}};
    \node[xshift=5.5cm] {$M$}
      child {node[xshift=-.4cm] {1}
        child {node {$y$}
          child[visible on=<3->, xshift=-.3cm] {node {$302$}
            child {node {$\{\{1 ↦ h₂\}\}$}}}
          child[visible on=<6->, xshift=.3cm] {node {$301$}
            child {node {$\{\{1 ↦ h₁\}\}$}}}}}
      child {node[xshift=.4cm] {2}
        child {node {$y$}
          child[visible on=<7->] {node {$301$}
            child {node[yshift=-2em] {$\{\{1 ↦ h₁, 2 ↦ h₂\}\}$}}}}};
  \end{tikzpicture}
\end{frame}

\begin{frame}[fragile]
  \frametitle{Adding a Hypothesis: Example}

  \begin{block}{New Goal}
    \begin{leancode}
      h₂ : R 301 302
      h₁ : R 300 301
      h₃ : R 300 302
    \end{leancode}
  \end{block}

  \pause

  To check whether $h₃$ enables more forward rules:

  \emph{Reuse} constructed state and \emph{add} $h₃$.
\end{frame}

\begin{frame}
  \frametitle{Adding a Hypothesis: Algorithm}

  Given a new hypothesis $h : T$, \\
  for each slot $s$ such that $T$ matches the type of $s$:
  \begin{enumerate}
    \item Add $h$ to $H(s, v, e)$ for each shared variable $v$.
    \item Get the matches in slot $s-1$ that agree with $h$ on the shared
          variables.
          Extend these matches with $h$ and add them to $M(s, v, e)$ for each $v$.
    \item For each extended match $m$, get the hypotheses in slot $s + 1$ that
          agree with $m$. Extend the matches again and add them to $M$.
    \item Repeat with slots $s + 2$, $s + 3$, \dots
    \item If a match is completed, report it as a new hypothesis.
  \end{enumerate}
\end{frame}

\begin{frame}[fragile]
  \frametitle{Benchmark Setup}

  \begin{leancode}
    @[aesop safe forward]
    axiom trans : ∀ x y z, R x y → R y z → R x z

    example
      (h₁ : R 300 301)
      (h₂ : R 301 302)
      ...
      (hₙ : R (300 + n - 1) (300 + n)) :
      True
  \end{leancode}
\end{frame}

\begin{frame}
  \frametitle{Benchmark Results}

  \medskip
  \begin{center}
    \begin{tabular}{rrrr}
      \blue{$n$} & \orange{$k$} & \purple{$T_{\text{old}}$} & \green{$T_{\text{new}}$} \\
      \toprule
      10  &  45 & \SI{1}{m}  & \SI{55}{ms} \\
      20  & 190 & \SI{18}{m} & \SI{210}{ms} \\
      30  & 435 & \text{?}   & \SI{470}{ms} \\
      40  & 780 & \text{?}   & \SI{860}{ms}
    \end{tabular}
  \end{center}

  \begin{tabular}{ll}
    \blue{$n$} & number of initial hypotheses \\
    \orange{$k$} & number of hypotheses added by forward reasoning \\
    \purple{$T_{\text{old}}$} & time taken by old implementation \\
    \green{$T_{\text{new}}$} & time taken by new implementation
  \end{tabular}
\end{frame}

\begin{frame}
  \centering{\Huge{\blue{Thanks!}}}
  % TODO Links/QR codes for these slides, Aesop repo, my CV
\end{frame}

\end{document}
